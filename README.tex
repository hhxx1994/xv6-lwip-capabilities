\documentclass[11pt]{article}

\usepackage{amsmath, amsfonts,amssymb,amsthm}
\usepackage{fullpage}

\title{6.828 Project Writeup}
\author{Patrick Hulin, Steven Valdez}
\begin{document}
\maketitle

\section{Overview}

\subsection{Goal}
The goal of this project was to implement a Capability model similar to the one in the Capsicum paper in xv6. In order to show Capability in action, we also ported a networking stack over into xv6.

\subsection{Networking}
To implement networking, we had to port both a ethernet driver and a TCP/IP network stack. For the driver, we ended up modifying a non-working port of the NE2K\_PCI driver by Shintaro Sheki, which we ended up having to fix to work with the memory model in the version of xv6 we were using. In addition, we ended up porting over lwIP, a lightweight TCP/IP stack commonly used for embedded systems. As a starting point, we used the sys\_arch bridge by Henry Hu. Unfortunately, due to various inconsistencies with the semantics used by lwIP and the version of xv6 we were developing on, a number of reduncancies were created that we didn't have time to solve, such as the thread and process relationship, and the sharing of page tables. Unlike JOS, which has a builtin system for IPC, we had to use a semaphore/mailbox system for communication between the networking code and other user code. In addition, we had to make sure that the network code, which was running at the user level, could receive memory from other processes, and as such ended up using a modified page table to give it access to some of the required memory.

Due to the lack of a physical environment, we ended up using QEMU's builtin features, such as network redirection to provide the bridge between the host machine, and the guest network adapter and system. 

\end{document}
